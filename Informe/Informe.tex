
\documentclass[12pt]{article}
\usepackage[english]{babel}
\usepackage[utf8x]{inputenc}
\usepackage{amsmath}
\usepackage{graphicx}
\usepackage[colorinlistoftodos]{todonotes}
\usepackage{subfig}
\usepackage{makecell}


\begin{document}
\begin{titlepage}
\newcommand{\HRule}{\rule{\linewidth}{0.5mm}} 
\center 
%opening

\textsc{\LARGE ITBA}\\[1.5cm] 
\textsc{\Large Sistemas Operativos}\\[0.5cm] 
\textsc{\large Profesores: Ariel Godio }\\[0.5cm] 

\HRule \\[0.4cm]
{ \huge \bfseries Trabajo Pr\'actico 2
\\\
Construcci\'on del N\'ucleo de un \\
Sistema Operativo }\\[0.4cm] 
\HRule \\[1.5cm]
 
\begin{minipage}{0.4\textwidth}
\begin{flushleft} \large
\emph{Alumnos:}\\
Lautaro Pinilla \\
Micaela Banfi \\
Joaqu\'in Battilana \\
Tom\'as Dorado \\
\end{flushleft}
\end{minipage}
~
\begin{minipage}{0.4\textwidth}
\begin{flushright} \large
\emph{} \\
57504 \\
57293 \\
57683 \\
56594 \\
\end{flushright}
\end{minipage}\\[2cm]


{\large 13 de Mayo de 2019}\\[2cm]

\vfil
\end{titlepage}

\section{Introducci\'on}
En el siguiente trabajo pr\'actico se desarrollar\'a la propia creaci\'on de un Kernel simple, en base al trabajo pr\'actica de la materia Arquitectura de las Computadoras. Se implementar\'an
mecanismos de IPC, Memory Management y Scheduling.

\section{Decisiones tomadas durante el desarrollo}

\subsection{Memory Manager}
\subsection{Scheduler}
\subsection{IPC}

\section{Limitaciones}

\section{Problemas encontrados durante el desarrollo y su solucion}
\section{Instrucciones de compilaci\'on y ejecuci\'on}
\section{Citas de fragmentos de código reutilizados de otras fuentes} 

\end{document}